En la sociedad actual, cada vez más procesos y dominios están incorporando técnicas de aprendizaje automático y visión artificial, en especial para la clasificación de características faciales (por ejemplo, el reconocimiento de género usado en sistemas de videovigilancia, o demoscópica en redes sociales). Pese a todas sus aplicaciones beneficiosas, existen ciertos temores asociados a los sesgos subyacentes a estas técnicas, ya sea por la propagación de sesgos existentes en la sociedad, o las dudas concernientes al diseño de los algoritmos y datos. Es por tanto crucial abordar las distintas fuentes de sesgo para conseguir resultados verdaderamente objetivos.

Este trabajo se enfoca en el problema del sesgo causado por desbalanceo de clases, cuya mitigación puede resolverse con técnicas de perfil algorítmico, o a nivel de datos. Se han detectado carencias en la bibliografía revisada, en cuanto a que faltan estudios que comparen ambos tipos de técnicas, y determinen cuál es más adecuado. Aparte, se quiere saber si las condiciones concretas de desbalanceo y complejidad de los datos influyen en la elección de un método mitigante. Todas estas brechas bibliográficas se han condensado en forma de tres preguntas de investigación. Para resolverlas, se han diseñado varios experimentos en los que se contrastan diferentes métodos de la bibliografía aplicados a dos datasets: UTKFace y PlantVillage (de características y complejidad diferentes), sobre los que se aplican diferentes escenarios de desbalanceo. Además, se propone un método novedoso de mitigación del desbalanceo a nivel de algoritmo, llamado Loss Focal Difusa. Los resultados obtenidos muestran la elevada eficacia de los métodos de sobremuestreo con generación sintética por medio de redes generativas antagónicas (GAN), en su variante \textit{Wasserstein} GAN, frente a los métodos de mitigación a nivel algorítmico. Entre éstos, el método nuevo propuesto obtiene las mejores métricas.

Con todo, los métodos algorítmicos pueden resultar más prácticos y rápidos de aplicar cuando la complejidad del problema no es muy elevada, lo cual se ha cuantificado mediante métricas de distancias y entropía de imágenes. Finalmente, los resultados experimentales sugieren que, según las prioridades del investigador, la aplicación de estos métodos puede ser la única vía para mitigar el desbalanceo, pues su tiempo de entrenamiento es menor que los de las GAN, y no corren el riesgo de generar características falsas en muestras artificiales.

% Además, en otros casos la aplicación de métodos algorítmicos puede ser la única vía para mitigar el desbalanceo, por ejemplo cuando la generación sintética de muestras puede conducir al aprendizaje de características falsas. Dentro de esta familia, la Loss Focal Difusa  (propuesta en este estudio) obtiene las mejores métricas experimentales cuando se la compara con otros métodos de la bibliografía.

\palabrasclave{Aprendizaje automático, clasificación en imágenes, mitigación de sesgo, desbalanceo de clases, redes neuronales, lógica difusa}
