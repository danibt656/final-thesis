Este Trabajo Fin de Grado culmina mi paso de cuatro años por la Escuela Politécnica Superior. Quiero expresar mi más sincero agradecimiento a todas las personas que me han ayudado a llegar a este momento en el que finaliza mi etapa en esta Universidad.

En primer lugar, me gustaría agradecer a mis padres y abuelos por su apoyo incondicional. Gracias por ser una fuente constante de inspiración y motivación en cada paso de este proceso. Sin su guía y apoyo, nunca habría llegado hasta aquí.

Asimismo, no puedo dejar de mencionar a mi tutora, Lara Quijano, por su orientación, su dedicación y su experiencia en el campo de la investigación. Gracias por su paciencia y por compartir conmigo sus conocimientos y consejos, los cuales me han ayudado a crecer tanto académica como personalmente. También quiero agradecer a mi profesora de física de secundaria, Gloria Moro, por infundirme confianza a la hora de afrontar retos difíciles e introducirme al campo de la ciencia y la tecnología.

Quiero mencionar también a aquellas personas que, sin formar parte directa de este proyecto, han sabido transmitirme su confianza y apoyo a lo largo de estos meses. Gracias a Héctor, Sergio, Pablo, Tracy, Carlos, David y Dani por estar siempre ahí.

Por último, quiero recordar al Dr. Richard Feynman (1918-1988), el célebre docente y astrofísico cuya filosofía de aprendizaje y humildad trato de aplicar cada día.