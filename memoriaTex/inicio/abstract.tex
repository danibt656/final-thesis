In today's society, an increasing number of processes and domains are incorporating machine learning and computer vision techniques, particularly for facial feature classification (e.g., gender recognition used in surveillance systems or demographic analysis in social networks). Despite their beneficial applications, there are concerns associated with the biases inherent in these techniques, whether due to the propagation of existing biases in society or doubts regarding algorithm and data design. Therefore, it is crucial to address the various sources of bias to achieve truly objective results.

This work focuses on the problem of bias caused by class imbalance, which can be mitigated using algorithmic or data-level techniques. Some gaps have been identified in the reviewed literature, as there is a lack of studies comparing both types of techniques and determining which is more suitable. Additionally, it is of great interest to investigate whether specific conditions of imbalance and data complexity influence the choice of a mitigation method. These gaps in the literature have been condensed into three research questions. To address them, several experiments have been designed to compare different methods from the literature applied to two datasets: UTKFace and PlantVillage, which have different characteristics and complexity. Different imbalance scenarios are applied to these datasets. Furthermore, a novel algorithm-level mitigation method called Diffuse Focal Loss is proposed. The results obtained demonstrate the high efficacy of synthetic oversampling methods using Generative Adversarial Networks (GANs), specifically the \textit{Wasserstein} GAN variant, compared to algorithm-level mitigation methods. Among these methods, the proposed novel approach achieves the best metrics.

However, algorithmic methods may be more practical and faster to apply when the complexity of the problem is not very high, as quantified by distance metrics and image entropy. Finally, the experimental results suggest that, depending on the researcher's priorities, the application of these methods may be the only way to mitigate class imbalance, as their training time is way shorter than that of GANs and they do not risk generating false features in synthetic samples.

\keywords{Machine learning, image classification, bias mitigation, class imbalance, neural networks, fuzzy logic}