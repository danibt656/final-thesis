En la sociedad, cada vez son más los procesos que pueden automatizarse gracias a la Inteligencia Artificial (IA), empleando algoritmos basados en aprendizaje automático \cite{vallimeena2019cnn}. Sin embargo, al aplicarse a los datos del mundo real, estos algoritmos pueden verse afectados por diversas fuentes de sesgo, entre las que pueden encontrarse la insuficiente representatividad de una clase con respecto a otra \cite{vluymans2019dealing}, el etiquetado poco preciso \cite{kafkalias2022bias}, el solapamiento ocasional entre clases distintas \cite{aka2021measuring}, o la desproporción en la cantidad de información que aportan ciertas muestras con respecto a otras \cite{li2019learning}. A causa de esto, en ocasiones las ventajas de la IA pueden verse eclipsadas por el temor a que los sesgos presentes se reproduzcan y propaguen en la sociedad \cite{nelson2019bias}. Es por ello que es imperativo ajustar los algoritmos y datasets utilizados en el aprendizaje para minimizar al máximo todas las posibles fuentes de sesgo. Este proyecto está orientado a ese objetivo, donde, dada la imposibilidad de abarcar todas las fuentes de sesgo en un único trabajo, se opta por estudiar en profundidad el sesgo causado por el \textit{desbalanceo de clases} \cite{liu2017fuzzy}.

El desbalanceo de clases en un dataset se define como una situación en la que una o varias clases se encuentran infrarrepresentadas numéricamente en relación a otra clase (o conjunto de clases) \cite{johnson2019survey}. Los motivos detrás del desbalanceo pueden ser muchos: el grupo minoritario está infrarrepresentado en la vida real (por ejemplo, las desigualdades de género presentes en muchos ámbitos) \cite{tianyu2018human}; es imposible recolectar el mismo número de muestras para cada clase (pensemos en las radiografías de enfermedades raras) \cite{rahimzadeh2020modified}; o incluso el desconocimiento o equivocación de aquellos encargados de recolectar y anotar los datos \cite{kafkalias2022bias}. Sea cual sea la causa, ante la aparición de este sesgo los diseñadores de algoritmos de IA deben tratar de reconocerlo, y actuar de acuerdo a ello.

De entre los dominios que se ven afectados por esta problemática, el reconocimiento facial (dentro del campo de la visión artificial) ocupa un lugar importante, dado el crecimiento de estos modelos en campos como la psicología, la interacción con vehículos autónomos, o los sistemas de videovigilancia policiales \cite{agbo2019face,han2020toward}. Generar modelos fiables de clasificación de imágenes faciales no es tarea sencilla, y el rendimiento de dichos modelos se ve condicionado por la existencia de conjuntos de datos suficientemente grandes y diversos \cite{agbo2019face} que representen todas las etiquetas que se pretende cubrir (género, raza, edad...). Más en detalle, la clasificación de género, que pretende discernir el género biológico\footnote{No se debe confundir esta clasificación con la distinción del \textit{género percibido}, ya que la naturaleza subjetiva de éste lo convertiría en un problema totalmente distinto. De hecho, sería más apropiado referirse al problema como ``clasificación de sexo'' y no de ``género'', pero en este trabajo se opta por lo segundo, ya que es la terminología usada en la literatura especializada.} de una persona basándose en distintas características faciales, ha cobrado popularidad en la última década por sus múltiples aplicaciones, sobre todo en sistemas de videovigilancia o estudios demoscópicos en redes sociales \cite{poornima2021classification,nelson2019bias}. Por ejemplo, los anuncios que provocan la identificación con el género propio son más probables de tener un impacto favorable en la imagen de la marca desde la perspectiva de los consumidores \cite{wu2020gender}. Mitigar los sesgos presentes en estos algoritmos es esencial no sólo para lograr una mejora técnica en el rendimiento de los modelos, sino también para trasladar esa idea de igualdad que se persigue en la sociedad.

Tras una extensa revisión bibliográfica centrada en el sesgo causado por el desbalanceo de clases, se ha observado que no existe un consenso entre los autores respecto a cómo tratarlo. Las diversas metodologías \cite{wadsworth2018achieving,vowels2020nestedvae,yan2019joint,hong2021fuzzy,tuncc2020fuzzy} pueden agruparse en dos tipos de estrategias a grandes rasgos: i) aquellas que atacan el problema del desbalanceo alterando los conjuntos de datos \cite{upadhyay2021state,nafi2020addressing,johnson2019survey}, y ii) aquellas que lo hacen modificando los algoritmos de aprendizaje \cite{lin2017focal,khan2017cost,narayanan2020transfer,rahimzadeh2020modified}. Pero, a día de hoy, faltan comparativas concluyentes entre ambos tipos de estrategias (de datos y algorítmicas), limitándose la mayoría de revisiones \cite{johnson2019survey,upadhyay2021state,vluymans2019dealing} a poner en contraste métodos del mismo tipo \cite{nafi2020addressing}. Esta falta de comparativas entre ambos tipos de estrategias puede deberse a la heterogeneidad de los entornos en los que se trabaja cada técnica (clasificaciones de imágenes faciales, de plantas, de radiografías...), pero también a la falta de puntos en común como serían el aplicar cada técnica a un mismo conjunto de datos. Aparte, las distintas publicaciones recogidas en las revisiones a menudo registran métricas distintas \cite{upadhyay2021state,nafi2020addressing,johnson2019survey,lin2017focal,khan2017cost,narayanan2020transfer,rahimzadeh2020modified}, con lo que es complicado determinar a priori qué método es objetivamente mejor. En general, hay cierta confusión con respecto a si existen metodologías aplicables a múltiples escenarios, o son necesarias soluciones \textit{ad hoc}.

Por todo ello, en este Trabajo Fin de Grado se pretende hacer esa comparación entre técnicas de los dos tipos sobre un ``punto en común'', que permita establecer pautas para atacar el sesgo por desbalanceo de clases en función de cada escenario de forma más completa. Así, al comienzo de esta investigación se formulan las siguientes hipótesis:

\begin{description}
    \item[Hipótesis 1:] Es posible mejorar \textit{las bases de datos} de imágenes usadas en problemas de clasificación para mitigar el sesgo por desbalanceo de clases.
    \item[Hipótesis 2:] De la misma forma, es posible mejorar \textit{los algoritmos} de clasificación de imágenes para el mismo fin.
\end{description}

Para responder a estas hipótesis se han formulado las siguientes preguntas de investigación (\textit{Research Questions}, en adelante RQ):

\begin{description}
    \item[RQ1:] Para mitigar el sesgo por desbalanceo, ¿es siempre mejor atacar los datos, los algoritmos, o depende de cada situación?
    
    \item[RQ2:] ¿Es necesario conocer de antemano si el conjunto de datos presenta desbalanceo y, en su caso, el grado de desbalanceo, para determinar el método a aplicar?
    
    \item[RQ3:] ¿Cómo afecta la complejidad del problema a la toma de decisiones sobre el mejor método de mitigación del desbalanceo a aplicar?
\end{description}

Para responder a estas preguntas, se ha realizado una comparación entre las diferentes estrategias identificadas en la bibliografía a nivel de algoritmos y de datos. Además, de forma novedosa, se propone un método de mitigación a nivel de algoritmo, denominado \textit{Loss Focal Difusa}, que se ha comparado junto con el resto de métodos. Por último, para dotar de algo más de generalidad a las conclusiones, se ha tratado de extender esta comparativa a otros dominios de la clasificación de imágenes aparte del de reconocimiento de género facial, pero manteniéndonos en el dominio de clasificación binaria. Por ello, se han empleado dos datasets: uno de género en imágenes y otro de infecciones en plantas de tomate.

El desarrollo de este Trabajo Fin de Grado ha conducido finalmente a las siguientes contribuciones:

\begin{description}
    \fontsize{11pt}{12pt}\selectfont
    \item[1:] Se aporta una comparativa en común de estrategias de mitigación del desbalanceo de clases, tanto a nivel de algoritmo como de datos.

    \item[2:] Se estudia el impacto de la complejidad del dominio del problema a la hora de dirimir la adecuación de una familia de métodos u otros.
    
    \item[3:] Se propone un método algorítmico propio, que mezcla conceptos de lógica difusa con funciones de pérdida adaptadas al desbalanceo, llamado Loss Focal Difusa.
\end{description}

El resto de este trabajo se ordena de la siguiente forma: el \textsc{Capítulo \ref{CAP:STATEOFART}} realiza una revisión del estado del arte más relevante de acuerdo a los objetivos de este proyecto, y justifica la formulación de las preguntas de investigación. El \textsc{Capítulo \ref{CAP:METODO}} expone cada una de las técnicas comparadas en los experimentos (tanto las estrategias de datos como las algorítmicas), así como el funcionamiento del método algorítmico propio, la Loss Focal Difusa. En el \textsc{Capítulo \ref{CAP:EXPERIMENTACION}} se presenta el diseño de cada uno de los experimentos y pruebas llevadas a cabo, así como los datasets sobre los que se realizan. Los resultados de dichos experimentos se discuten y comparan en el \textsc{Capítulo \ref{CAP:RESULTS}}. Para terminar, el \textsc{Capítulo \ref{CAP:CONCLUSION}} expone las conclusiones extraídas de este proyecto de investigación, y las posibles líneas de investigación futura.