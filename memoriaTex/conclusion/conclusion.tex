En este capítulo se exponen las conclusiones finales, y se comentan posibles líneas de trabajo futuro que podrían seguirse para expandir el alcance de la presente investigación.

%%%%%%%%%%%%%%%%%%%%%%%%%%%%%%%%%%%%%%%%%%%%%%%%%%%%
\section{Conclusiones\label{SEC:CONCLUSIONES}}
%%%%%%%%%%%%%%%%%%%%%%%%%%%%%%%%%%%%%%%%%%%%%%%%%%%%

El reconocimiento de género en imágenes es un campo del aprendizaje automático que tiene aplicaciones beneficiosas para la sociedad, como la mejora de la seguridad, la personalización de experiencias \textit{online} y la identificación de desigualdades de género en ciertos ámbitos. Por ello, es crucial abordar las distintas fuentes de sesgo en los algoritmos utilizados en este campo. En esta investigación en concreto se ha enfocado en el sesgo por desbalanceo de clases.

La revisión bibliográfica \cite{johnson2019survey,aka2021measuring,kim2019multiaccuracy,krishnan2020understanding,li2019learning} ha revelado que el desbalanceo de clases puede abordarse de dos formas: mediante el equilibrado de las distribuciones de los datasets, o mediante la adaptación de los algoritmos clasificadores para prestar más atención a las clases minoritarias. Se han identificado una serie de vacíos en la literatura (\textit{research gaps}), ya que no se han encontrado comparativas concluyentes sobre cuál es el método algorítmico más efectivo, y tampoco hay comparativas que pongan en contraste ambas estrategias para determinar cuál es más apropiada, y en qué situaciones. Para abordar estos asuntos, se han formulado una serie de preguntas de investigación, y a fin de responderlas, se han elaborado experimentos que pusieran en comparación métodos de ambas categorías sobre dos datasets, en diferentes escenarios de desbalanceo.

Los resultados experimentales parecen indicar que la utilización de WGAN (un método de mitigación a nivel de datos) da la mayor mejora en el reconocimiento de la clase minoritaria, con una mejora de hasta un 21\% en F1-score (y un 2\% en G-mean) frente al mejor método algorítmico, la Loss Focal Difusa, cuando existe desbalanceo. No obstante, entrenar la red generativa implica añadir hasta 36 horas al tiempo de entrenamiento del clasificador. Además, existe el riesgo de generar ``información falsa'' que, según el dominio de aplicación, puede ser inaceptable. En estos escenarios ganan peso los métodos algorítmicos, que además no alteran la distribución de los datos, por lo que los tiempos de ejecución no aumentan.

Para terminar, se ha estudiado el papel que desempeña la complejidad del problema en la elección del método de mitigación del desbalanceo, por medio de comparar los experimentos sobre un dataset ``sencillo'' (PlantVillage) y otro más ``complejo'' (UTKFace). Dicha noción de complejidad se ha cuantificado mediante métricas como la distancia de Minkowski y las entropías de Shannon y GLCM. En resumen, bajo escenarios de alta complejidad, es recomendable emplear el aumento de datos mediante WGAN, siempre y cuando se disponga de una base lo suficientemente amplia para permitir que el generador aprenda a producir muestras aceptables. Por otro lado, en datasets de baja complejidad o donde la generación sintética no sea una alternativa viable, puede ser mejor idea aplicar métodos algorítmicos para abordar el desbalanceo (y si la complejidad es reducida, se seguirán obteniendo métricas elevadas). Aquí, de nuevo, se puede emplear la Loss Focal Difusa.


%%%%%%%%%%%%%%%%%%%%%%%%%%%%%%%%%%%%%%%%%%%%%%%%%%%%
\section{Trabajo futuro\label{SEC:FUTUREWORK}}
%%%%%%%%%%%%%%%%%%%%%%%%%%%%%%%%%%%%%%%%%%%%%%%%%%%%

Como primera línea de posible investigación futura, y dado que su rendimiento ha resultado exitoso dentro de los métodos mitigantes a nivel de datos, se puede explorar la optimización del FCS de la Loss Focal Difusa. En el campo de la optimización de sistemas difusos, \citet{alcala1999techniques} revisa diversos tipos de técnicas basadas en algoritmos genéticos \cite{kang2006optimization} o redes neuronales \cite{jang1993anfis}. También destaca el método PSO de \citet{esmin2002particle}. Una investigación futura podría explorar la capacidad de alguno de estos mecanismos para optimizar la base de reglas y/o las funciones de pertenencia del FCS propuesto.

Se ha hablado además de la complejidad de los datasets, y de cómo las métricas de género facial podrían mejorarse si se aumentara la densidad de muestras, por lo que podrían repetirse los experimentos para todos los métodos estudiados, con otro dataset de imágenes faciales más grande que UTKFace.

En cuanto a la WGAN, se podría evaluar de manera más detallada para determinar si existe un límite en el grado de desbalanceo, a partir del cual no puede aprender a generar muestras con un nivel de realismo suficiente. Comprender este límite es esencial para establecer expectativas realistas sobre las capacidades del sobremuestreo sintético generativo. Otro aspecto importante de esta técnica es que requiere un dataset mínimo para entrenar. Así, una posible idea de investigación futura sería examinar la capacidad de los modelos como \textit{Stable Diffusion} \cite{rombach2022high} o \textit{Midjourney} \cite{oppenlaender2022creativity} para generar datasets completamente nuevos, y comparar su desempeño con las GAN. Este enfoque podría proporcionar una alternativa interesante en situaciones en las que apenas haya muestras de la clase a aumentar.

Por último, sería relevante explorar la aplicabilidad de los métodos más destacados en este trabajo en escenarios de desbalanceo múltiple o multiclase. Por ejemplo, considerando la etnia o la edad además del género en imágenes faciales. Estudiar la efectividad y las consideraciones específicas de estas técnicas en situaciones de desbalanceo más complejas ampliaría su aplicabilidad a una gama más amplia de problemas.