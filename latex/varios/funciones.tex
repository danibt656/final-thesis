En este apéndice se presenta un listado de funciones y otro de los entornos. Se presentan en formato de tabla a modo de resumen y para que puedan ser impresas a modo de referencia. En cada función o entorno se indica si debe usarse en el cuerpo del texto o en el preámbulo, los parámetros opcionales si tiene y los obligatorios si tiene. También se presentan aquellos comandos de \LaTeXe\ que no sólo ha sido modificado su comportamiento sino que también han cambiado los parámetros.

\def\funcl#1#2#3{\textbf{\textbackslash#1}&#2&#3\\}
\def\entorn#1#2#3{\textbf{#1}&#2&#3\\}
\def\Bttl#1{\multicolumn{3}{c}{\small\textbf{#1}}\\}
\def\ttl#1{\hline\multicolumn{3}{|c|}{\textbf{#1}}\\\hline}
\def\myseparator{\multicolumn{3}{c}{\mbox{}}\\}

\section{Comandos en el preámbulo}
\begin{center}
\begin{scriptsize}
  \begin{longtable}{lll}
    \myseparator
    \ttl{Autoría}
    \funcl{advisor}{}{\{tutor\}}
    \funcl{author}{}{\{titulo\}}
    \funcl{coadvisor}{}{\{cotutor\}}
    \funcl{copyrightdate}{}{\{fecha\}}
    \funcl{faculty}{}{\{facultad/escuela\}}
    \funcl{levelin}{}{\{titulacion\}}
    \funcl{title}{[título corto]}{\{titulo\}}
    \funcl{speaker}{}{\{ponente\}}
    \myseparator
    \ttl{Decoraciones}
    \funcl{coverdata}{}{\{texto\}}
    \funcl{facultylogo}{}{\{fichero\}}
    \funcl{facultylogowide}{}{\{dimension\}}
    \myseparator
    \ttl{Directorios}
    \funcl{codesdir}{}{\{directorio\}}
    \funcl{datadir}{}{\{directorio\}}
    \funcl{graphicsdir}{}{\{directorio\}}
    \funcl{logosdir}{}{\{directorio\}}
    \myseparator
    \ttl{Prefacio}
    \funcl{abstractfile}{}{\{fichero\}}
    \funcl{ackfile}{}{\{fichero\}}
    \funcl{dedication}{}{\{dedicatoria\}}
    \funcl{famouscite}{}{\{cita\}}
    \funcl{keywords}{}{\{palabras\}}
    \funcl{palabrasclave}{}{\{palabras\}}
    \funcl{prefacefile}{}{\{fichero\}}
    \funcl{privateaddress}{}{\{direccion\}}
    \funcl{resumenfile}{}{\{fichero\}}
  \end{longtable}
\end{scriptsize}
\end{center}
\newpage
\section{Commandos en el cuerpo del texto}
\begin{center}
\begin{scriptsize}
  \begin{longtable}{lll}
    \myseparator
    \ttl{Código}
    \funcl{Code}{[etiqueta]}{\{pie corto\}\{pie largo\}\{fichero\}\{línea inicial\}\{línea final\}\{numeración inicial\}\{lenguaje\}}
    \funcl{AdaCode}{[etiqueta]}{\{pie corto\}\{pie largo\}\{fichero\}\{línea inicial\}\{línea final\}\{numeración inicial\}}
    \funcl{ASMCode}{[etiqueta]}{\{pie corto\}\{pie largo\}\{fichero\}\{línea inicial\}\{línea final\}\{numeración inicial\}}
    \funcl{ASMMotorolaCode}{[etiqueta]}{\{pie corto\}\{pie largo\}\{fichero\}\{línea inicial\}\{línea final\}\{numeración inicial\}}
    \funcl{CCode}{[etiqueta]}{\{pie corto\}\{pie largo\}\{fichero\}\{línea inicial\}\{línea final\}\{numeración inicial\}}
    \funcl{CPPCode}{[etiqueta]}{\{pie corto\}\{pie largo\}\{fichero\}\{línea inicial\}\{línea final\}\{numeración inicial\}}
    \funcl{CSharpCode}{[etiqueta]}{\{pie corto\}\{pie largo\}\{fichero\}\{línea inicial\}\{línea final\}\{numeración inicial\}}
    \funcl{GnuplotCode}{[etiqueta]}{\{pie corto\}\{pie largo\}\{fichero\}\{línea inicial\}\{línea final\}\{numeración inicial\}}
    \funcl{HaskellCode}{[etiqueta]}{\{pie corto\}\{pie largo\}\{fichero\}\{línea inicial\}\{línea final\}\{numeración inicial\}}
    \funcl{HTMLCode}{[etiqueta]}{\{pie corto\}\{pie largo\}\{fichero\}\{línea inicial\}\{línea final\}\{numeración inicial\}}
    \funcl{JavaCode}{[etiqueta]}{\{pie corto\}\{pie largo\}\{fichero\}\{línea inicial\}\{línea final\}\{numeración inicial\}}
    \funcl{LaTeXCode}{[etiqueta]}{\{pie corto\}\{pie largo\}\{fichero\}\{línea inicial\}\{línea final\}\{numeración inicial\}}
    \funcl{LispCode}{[etiqueta]}{\{pie corto\}\{pie largo\}\{fichero\}\{línea inicial\}\{línea final\}\{numeración inicial\}}
    \funcl{MakeCode}{[etiqueta]}{\{pie corto\}\{pie largo\}\{fichero\}\{línea inicial\}\{línea final\}\{numeración inicial\}}
    \funcl{MathematicaCode}{[etiqueta]}{\{pie corto\}\{pie largo\}\{fichero\}\{línea inicial\}\{línea final\}\{numeración inicial\}}
    \funcl{MatlabCode}{[etiqueta]}{\{pie corto\}\{pie largo\}\{fichero\}\{línea inicial\}\{línea final\}\{numeración inicial\}}
    \funcl{OctaveCode}{[etiqueta]}{\{pie corto\}\{pie largo\}\{fichero\}\{línea inicial\}\{línea final\}\{numeración inicial\}}
    \funcl{PascalCode}{[etiqueta]}{\{pie corto\}\{pie largo\}\{fichero\}\{línea inicial\}\{línea final\}\{numeración inicial\}}
    \funcl{PerlCode}{[etiqueta]}{\{pie corto\}\{pie largo\}\{fichero\}\{línea inicial\}\{línea final\}\{numeración inicial\}}
    \funcl{PHPCode}{[etiqueta]}{\{pie corto\}\{pie largo\}\{fichero\}\{línea inicial\}\{línea final\}\{numeración inicial\}}
    \funcl{PythonCode}{[etiqueta]}{\{pie corto\}\{pie largo\}\{fichero\}\{línea inicial\}\{línea final\}\{numeración inicial\}}
    \funcl{RCode}{[etiqueta]}{\{pie corto\}\{pie largo\}\{fichero\}\{línea inicial\}\{línea final\}\{numeración inicial\}}
    \funcl{RubyCode}{[etiqueta]}{\{pie corto\}\{pie largo\}\{fichero\}\{línea inicial\}\{línea final\}\{numeración inicial\}}
    \funcl{ScilabCode}{[etiqueta]}{\{pie corto\}\{pie largo\}\{fichero\}\{línea inicial\}\{línea final\}\{numeración inicial\}}
    \funcl{SQLCode}{[etiqueta]}{\{pie corto\}\{pie largo\}\{fichero\}\{línea inicial\}\{línea final\}\{numeración inicial\}}
    \funcl{VHDLCode}{[etiqueta]}{\{pie corto\}\{pie largo\}\{fichero\}\{línea inicial\}\{línea final\}\{numeración inicial\}}
    \funcl{XMLCode}{[etiqueta]}{\{pie corto\}\{pie largo\}\{fichero\}\{línea inicial\}\{línea final\}\{numeración inicial\}}
    \myseparator
    \ttl{Ecuaciones}
    \funcl{boxed}{}{\{ecuación\}}
    \myseparator
    \ttl{Estructura}
    \funcl{cleardoublepage}{}{}
    \funcl{part}{[título corto]}{\{titulo\}\{fichero\}}
    \funcl{part}{[título corto]}{\{titulo\}}
    \funcl{chapter}{[título corto]}{\{titulo\}\{fichero\}}
    \funcl{chapter}{[título corto]}{\{titulo\}}
    \funcl{section}{[título corto]}{\{titulo\}\{fichero\}}
    \funcl{section}{[título corto]}{\{titulo\}}
    \funcl{subsection}{[título corto]}{\{titulo\}\{fichero\}}
    \funcl{subsection}{[título corto]}{\{titulo\}}
    \funcl{subsubsection}{[título corto]}{\{titulo\}\{fichero\}}
    \funcl{subsubsection}{[título corto]}{\{titulo\}}
    \funcl{paragraph}{[título corto]}{\{titulo\}\{fichero\}}
    \funcl{paragraph}{[título corto]}{\{titulo\}}
    \funcl{subparagraph}{[título corto]}{\{titulo\}\{fichero\}}
    \funcl{subparagraph}{[título corto]}{\{titulo\}}
    \myseparator
    \ttl{Figuras y tablas}
    \funcl{subfigure}{[etiqueta]}{\{pie\}\{contenido\}}
    \funcl{subtable}{[etiqueta]}{\{pie\}\{contenido\}}
    \myseparator
    \ttl{Gantt}
    \funcl{milestone}{[etiqueta enlace]}{\{etiqueta gantt\}\{tiempo\}}
    \funcl{taskbar}{[etiqueta enlace][porcentaje fin]}{\{etiqueta gantt\}\{tiempo inicio\}\{tiempo fin\}}
    \funcl{taskgroup}{[etiqueta enlace][porcentaje fin]}{\{etiqueta gantt\}\{tiempo inicio\}\{tiempo fin\}}
    \funcl{FtoFlink}{}{\{eetiqueta inicio\}\{eetiqueta fin\}}
    \funcl{FtoSlink}{}{\{eetiqueta inicio\}\{eetiqueta fin\}}
    \funcl{StoSlink}{}{\{eetiqueta inicio\}\{eetiqueta fin\}}
    \myseparator
    \myseparator
    \myseparator
    \ttl{Gráficas}
    \funcl{plotlined}{}{\{fichero datos\}\{titulo datos\}}
    \funcl{plotdatalined}{}{\{fichero datos\}\{titulo datos\}}
    \funcl{plotdata}{}{\{fichero datos\}\{titulo datos\}}
    \funcl{plotfunction}{}{\{fichero datos\}\{titulo datos\}}
    \myseparator
    \ttl{Imágenes}
    \funcl{image}{}{\{ancho\}\{alto\}\{fichero\}}
    \funcl{imageIL}{ancho}{fichero}
    \myseparator
    \ttl{Presupuestos}
    \funcl{budgettitle}{}{\{titulo presupuesto\}}
    \funcl{concept}{}{\{título\}\{precio unitario\}\{cantidad\}\{coste total\}}
    \funcl{separator}{}{}
    \funcl{subconcept}{}{\{título\}\{precio unitario\}\{cantidad\}\{coste total\}}
    \funcl{subtotal}{}{\{subtotal\}}
    \funcl{total}{}{\{total presupuesto\}}
    \myseparator
    \ttl{Texto citado}
    \funcl{onlinecitation}{}{\{autor\}\{texto\}}
  \end{longtable}
\end{scriptsize}
\end{center}

\section{Entornos}

De igual forma en la siguiente tabla se presentan los distintos entornos creados o modificados para este estilo. Todos los entornos deben usarse dentro del cuerpo del documento.

\begin{footnotesize}
  \begin{longtable}{lll}
    \ttl{General}
    \entorn{algorithm}{[pie corto]}{\{etiqueta\}\{pie completo\}}
    \entorn{algorithmN}{[pie corto]}{\{etiqueta\}\{pie completo\}}
    \entorn{budget}{}{}
    \entorn{equation}{[etiqueta]}{\{título\}}
    \entorn{figure}{[pie corto]}{\{etiqueta\}\{pie completo\}}
    \entorn{gantt}{}{\{tiempo inicio\}\{tiempo fin\}}
    \entorn{largecitation}{}{\{autor\}}
    \entorn{multiequation}{}{}
    \entorn{table}{[pie corto]}{\{etiqueta\}\{pie completo\}}
    \entorn{textbox}{[pie corto]}{\{etiqueta\}\{pie completo\}}
    \myseparator
    \ttl{Listados especiales}
    \entorn{functionality}{}{}
    \entorn{objetive}{}{}
    \entorn{functional}{}{}
    \entorn{nonfunctional}{}{}
    \entorn{simplelist}{}{}
    \myseparator
    \ttl{Gráficas}
    \entorn{gnuplot}{[opciones]}{}
    \entorn{loglogplot}{[posicion leyenda]}{\{título\}\{título eje x\}\{título eje y\}\{ancho\}\{alto\}}
    \entorn{semilogxplot}{[posicion leyenda]}{\{título\}\{título eje x\}\{título eje y\}\{ancho\}\{alto\}}
    \entorn{semilogyplot}{[posicion leyenda]}{\{título\}\{título eje x\}\{título eje y\}\{ancho\}\{alto\}}
    \entorn{xyplot}{[posicion leyenda]}{\{título\}\{título eje x\}\{título eje y\}\{ancho\}\{alto\}}
  \end{longtable}
\end{footnotesize}
